
\documentclass{beamer}
\usepackage[UTF8]{ctexcap}
\usepackage{xcolor}
\usetheme{EastLansing}
\begin{document}
\kaishu
\title[货币政策]{货币政策}
\subtitle{基于消息冲击}
\author{张敏 武汉大学研究生}
\institute[经济研究]{Amy}
\date[2017/10/30]{经济研究~2016年第2期}
\logo{\includegraphics[width=0.8cm,height=0.8cm]{wuda.jpg}}

\begin{frame}
\titlepage
\end{frame}

\begin{frame}{目录}
\fangsong
\small
\tableofcontents[hideallsubsections]
\end{frame}

\section{摘要}

\begin{frame}
\tableofcontents[currentsection,hideallsubsections]
\end{frame}

\begin{frame}{摘要}
\begin{itemize}
	\item 本文在新凯恩斯DSGE的框架下~~
	\vspace{2ex}
	\item 本文得出结论包括:
	\begin{itemize}
		\item 1.我货币政策预期冲击的效果要远强于未预期冲击;
			\vspace{2ex}
		\item 2.与美国相比, 我国货币政策的调控力
		度更大,持续性更小且存在轻微超调;
			\vspace{2ex}
		\item 3.我国货币政策的特征源于经济主体的短视预期以及央行货币政策的不连贯性。
	\end{itemize}
\end{itemize}
\end{frame}

\section{引言}
\begin{frame}
\tableofcontents[currentsection,hideothersubsections]
\end{frame}


\begin{frame}{消息冲击}


\begin{block}{消息冲击}
	\begin{itemize}
		\item 消息冲击的概念源于
		庇古, 由消息推动的经济周期又称为 庇古周期 ( Pigou, 1927)。
		\item 指人们在某一时刻提前获得有关未来发
		展状态的新信息。
	\end{itemize}
\end{block}

\begin{beamerboxesrounded}{政策性消息冲击}
	\begin{itemize}
		\item 在模型设定上, 政策性消息冲击表现为历史上已经预料到的并在本期得以实现
		的政策冲击部分
		\item 于是当期政策冲击就可以区分为历史上已经预料到的政策冲击( 亦即消息冲击),
		以及未预料到的其它外生冲击。本文分别简称为预期冲击和未预期冲击。
	\end{itemize}
\end{beamerboxesrounded}
\end{frame}

\begin{frame}{本文工作}
\begin{itemize}
	\item 本文首先构建一个同时包含预期与未预期货币政策冲击的典型三方程新凯恩
	斯 DSGE 模型作为基准模型, 模拟输出预期与未预期的货币政策冲击对我国通货膨胀的作用。
	\vspace{2ex}
	\item 随后, 通过比较中美两国两类冲击作用的异同, 以及替换参数和预期结构展开分析, 来探寻我国货币
	政策特征的根源及其机制, 再据此提出货币政策改进建议。
	\vspace{2ex}
	\item  为确保模型结果的稳健性, 本文分别考
	虑了不同的参数环境 区分非货币政策冲击的预期与未预期部分以及对基准模型进行扩展三种情况。
\end{itemize}
\end{frame}

\

\section{理论来源与研究进展}
\begin{frame}
\tableofcontents[currentsection,hideothersubsections]
\end{frame}

\subsection{消息冲击的理论来源与应用}



\begin{frame}{消息冲击已有研究}
\begin{block}{国内研究}
	\begin{itemize}
		\item 	国内目前只有少数关于消息冲击的研究应用。
		\begin{enumerate}
			\item 吴化斌等( 2011) 在新凯恩斯 DSGE 框架下研
			究了我国财政政策消息冲击的影响;
			\item 	庄子罐等( 2012) 在生产率 特定投资技术以及政府支出中引
			入消息冲击, 在新古典 DSGE 框架下考察了消息冲击对我国经济波动的作用
		\end{enumerate}
	\end{itemize}
\end{block}
\end{frame}

\subsection{我国货币政策的通货膨胀管理问题}

\begin{frame}{我国货币政策的通货膨胀管理问题}
2008 年金融危机后, 国内不少学者对我国货币政策的通货膨胀管理职能进行了研究
\end{frame}

\section{基准模型的构建}
\begin{frame}
\tableofcontents[currentsection,hideothersubsections]
\end{frame}

\subsection{经济主体行为}
\begin{frame}{经济主体行为}
content...
\end{frame}

\subsection{经济动态系统}
\begin{frame}{经济动态系统}
content...
\end{frame}


\section{参数估计和最优消息期限选择}
\begin{frame}
\tableofcontents[currentsection,hideothersubsections]
\end{frame}

\subsection{方法说明}
\begin{frame}{方法说明}
content...
\end{frame}



\subsection{参数校准}
\begin{frame}{参数校准}

\end{frame}

\subsection{消息期限选择和贝叶斯估计}
\begin{frame}{消息期限选择和贝叶斯估计}
content...
\end{frame}

\section{模型表现与实证结果}
\begin{frame}
\tableofcontents[currentsection,hideothersubsections]
\end{frame}

\subsection{预测优度比较}
\begin{frame}{预测优度比较}
\begin{figure}
	\centering
	\includegraphics[width=0.7\linewidth]{fig/yuceyoudu}
	\label{fig:yuceyoudu}
\end{figure}
\end{frame}


\subsection{脉冲响应分析 预期与未预期货币
	政策对通货膨胀的作用}

\begin{frame}{脉冲响应分析}
\begin{figure}
	\centering
	\includegraphics[width=0.7\linewidth]{fig/maichongyuceweiyuqi}
	\label{fig:maichongyuceweiyuqi}
\end{figure}

\end{frame}





\end{document}
